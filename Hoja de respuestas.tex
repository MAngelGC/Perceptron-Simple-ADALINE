\documentclass{article}
\usepackage{graphicx} % Required for inserting images
\usepackage{amssymb}
\usepackage{hyperref}
\hypersetup{
    colorlinks=true,
    linkcolor=black,
    filecolor=magenta,      
    urlcolor=blue,
    pdftitle={Hoja de respuestas - Práctica 2},
    pdfpagemode=FullScreen,
}
\renewcommand{\contentsname}{Índice}

\title{Modelos Computacionales \\ Hoja de respuestas - Práctica 2}
\author{Miguel Ángel González Caminero}
\date\noexpand

\begin{document}

\maketitle
\tableofcontents
\newpage

\section{¿Qué ocurre si damos un valor negativo de LR? Da una explicación interpretando gráficamente lo que ocurre.}

Si damos valores negativos a la tasa de aprendizaje, vemos como la línea que representa la neurona se va alejando cada vez más de los valores,
sin ajustarse a estos. Esto se debe a que la función de optimización sigue la dirección de descenso del gradiente. Si damos valores negativos a LR,
cambiamos el signo de esta, y por tanto nos alejamos del mínimo que se busca.

\end{document}